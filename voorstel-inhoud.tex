\section{Introductie} % The \section*{} command stops section numbering
\label{sec:introductie}

Dit onderzoek zal draaien rond het onderzoeken en vergelijken van verschillende tracking technologieën, zowel actieve als passieve, in verschillende use-cases. Dit zal worden gedaan in samenwerking met Auxcis. Dit bedrijf wil haar bestaande aanbod Track and Trace oplossingen, welke momenteel berusten op het gebruik van passieve RFID tags, uitbreiden met oplossingen gebaseerd op actieve technologieën (zoals BLE, CenTrak, UHF, GPS enz.). Voordat ze dit willen verwezelijken zijn zij geïnteresseerd in een onderzoek naar de voor- en nadelen, en toepassingsgebieden/use-cases van de verschillende actieve technologieën die realistisch door hun inzetbaar zijn.

Tegenwoordig wordt er steeds meer gebruik gemaakt van trackingtechnologiëen, echter is het in vele gevallen moeilijk te kiezen tussen de technologie waarvan gebruik zal gemaakt worden aangezien de specificaties verschillen tussen deze technologiëen. Het doel van dit onderzoek is dan ook het kiezen tussen deze verschillende opties gemakkelijker maken, door op een overzichtelijke manier deze verschillende opties met hun voor-en nadelen en toepassingsgebieden weer te geven.

Practisch gezien zullen deze verschillende toepassingsgebieden vooral onderscheden worden door het bereik (samenhangend met de grootte van het gebouw) en de nauwkeurigheid/detectiegraad (samenhangend met de grootte van het voorwerp, en de snelheid van verplaatsing) van de technologie, alsook het verschil tussen een statische en dynamische setup. Ook andere factoren zoals kost (Hardware, licence fees et.), het effect van meerdere verdiepingen en gemak van integratie in bestaande systemen zullen van belang zijn, en deze zullen dus ook opgenomen worden in het onderzoek en conclusies.

%---------- Stand van zaken ---------------------------------------------------

\section{State-of-the-art}
\label{sec:state-of-the-art}

\subsection{Actieve vs. passieve trackingtechnologieën}

Actieve trackers versturen voortdurend een signaal welke opgevangen kan worden door een basisstation, en zo kan in real-time de positie van de tracker (en bij uitbreiding het voorwerp of de persoon waaraan deze verbonden is) gelocaliseerd en gevolgd worden.
Dit in tegenstelling tot pasieve trackers, welke een locatie opslaan, die later kan uitgelezen worden om de locatiegeschiedenis te bepalen. \autocite{Rosenfeld2017}

\subsection{Waarom dit onderzoek?}
De keuze tussen de verschillende technologieën is niet gemakkelijk en zeer afhankelijk van de toepassing. Echter zijn er niet veel tot geen bronnen vindbaar die deze vergelijking maken. De bronnen die er zijn focussen zich voornamelijk op de voornaamste speler, GPS, gevolgd door satteliettracking en gebruik van het mobiel netwerk. Alhoewel dit zeer bruikbare technologieën zijn zijn deze niet geschikt voor de toepassing in gebouwen, aangezien zij niet nauwkeurig genoeg zijn en problemen hebben bij de aanwezigheid van verdiepingen. Andere meer besproken technologieën zijn BLE en WiFi, welke veelbelovend zijn, maar er bestaan geen vergelijkingen met andere minder bekende technologieën zoals CenTrak. Dit onderzoek is bedoeld ondersteuning te bieden bij deze keuze. \autocite{Deloitte2021} \autocite{Nijhawan2021}

%---------- Methodologie ------------------------------------------------------
\section{Methodologie}
\label{sec:methodologie}

Het onderzoek zal voornamelijk experimenteel verlopen, met testopstellingen van de fysieke hardware (opgesteld op het kantoor van Auxcis). De kwantitatieve resultaten (nauwkeurigheid, detectiegraad, bereik enz.) zullen hiermee worden bepaald. Voor het bepalen van het gemak van integratie in bestaande infrastructuur zal een server gebouwd worden met Dotnet, welke zal dienen als 'vertaler' tussen de fysieke infrastructuur en de bestaande servers van Auxcis. Alhoewel de resultaten op dit vlak voornamelijk een mening van mezelf zullen zijn (moeilijkheidsgraad is subjectief), zijn hier ook meetbare parameters zoals aantal werkuren en aantal lijnen code aan verbonden. Dus ook hier zijn kwantitatieve conclusies uit te trekken. Factor kost zal uiteraard vooral afhangen van de betaalde prijs van de opstelling, uiteraard gestandaardiseerd om alle technologieën hetzelfde te beoordelen.

%---------- Verwachte resultaten ----------------------------------------------
\section{Verwachte resultaten}
\label{sec:verwachte_resultaten}

Aangezien het doel van dit onderzoek een vergelijking maken is, zullen de resultaten grotendeels kunnen worden neergeschreven in een tabel, met een rij voor elke technologie en een kolom voor elke gemeten/bepaalde variabele. Ook zal aan elke technologie een score worden gegeven op basis van de geschiktheid voor een bepaalde toepassing, met hiervoor ook een boomdiagram voor het maken van een keuze van technologie.

%---------- Verwachte conclusies ----------------------------------------------
\section{Verwachte conclusies}
\label{sec:verwachte_conclusies}

Ik verwacht dat een klassieke technologie zoals GPS het slecht zal doen tegenover meer moderne en kleinschaligere technologieën met basisstations zoals BLE of CenTrak, door de slechtere nauwkeurigheid. Welke van de andere technologieën op welke vlakken zal uitschijnen zal afhangen van het onderzoek. Alhoewel ik denk dat CenTrak het goed zal doen aangezien dit voornamelijk is ontworpen voor de tracking van objecten binnen een ziekenhuis dus dit heeft zijn toepassing al mee.