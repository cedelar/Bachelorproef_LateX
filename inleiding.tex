%%=============================================================================
%% Inleiding
%%=============================================================================

\chapter{Inleiding}
\label{ch:inleiding}

\section{Probleemstelling}
\label{sec:probleemstelling}

Momenteel is Aucxis, de opdrachtgever en partner voor dit onderzoek, bezig met het uitwerken en ontwikkelen van haar Polaris platform. Het voornaamste doel van dit platform is het kunnen volgen van voorwerpen binnen en buiten bedrijven. Dit aan de hand van verschillende technologieën, voornamelijk RFID en BLE binnen het bedrijf en GPS daarbuiten. 
Het concept van dit platform is dat er logische locaties bestaan (zoals warenhuis 1 of keuken) waar de voorwerpen zich bevinden. Voor een goede werking van dit systeem is het duidelijk nodig dat uit de data, bekomen uit de hardware, kan afgeleid worden op welke logische locatie een voorwerp zich bevindt en wanneer het zich verplaatst tussen locaties. Dit is geen normaal softwareprobleem want logischerwijs zal de manier waarop de hardware staat opgesteld een grote invloed hebben op de nauwkeurigheid en correctheid van deze lokalisaties en verplaatsingen. Hierdoor is het van groot belang voor de correcte werking van dit platform dat er een goede opstelling wordt gekozen. Om dit te kunnen doen is er een onderzoek nodig naar wat deze optimale opstelling zou kunnen zijn. Dit is ook de reden dat dit onderzoek in het leven is geroepen.

\section{Wie is Aucxis}
\label{sec:wie-is-aucxis}
Aucxis is een bedrijf gevestigd te Stekene, welke actief is sinds 1983. Bij oprichting was het een bedrijf dat zich toespitste op het ontwikkelen van enerzijds de bewaring (Procescontrole) en anderzijds de veilinginfrastructuur (E-trade) voor groenten en fruit. Het werd hier vrij snel een toonaangevend bedrijf en werd wereldleider tegen 2000. Later, in 2007 richtten ze een 3e businessunit op, namelijk de RFID divisie, welke zich focust op het onderzoek naar en ontwikkeling van RFID toepassingen. Vandaag is het nog steeds een toonaangevend bedrijf met in-house oplossingen voor allerhande toepassingen binnen deze sectoren en is ze nog steeds actief bezig aan de optimalisering en uitbreiding hiervan. Recentelijk is dit bedrijf ook geïnteresseerd in het uitbreiden van haar praktijken richting BLE toepassingen.\autocite{Aucxis2020}

\section{Onderzoeksvraag}
\label{sec:onderzoeksvraag}

Zoals duidelijk is geworden in de probleemstelling is er nood aan een onderzoek voor het vinden van een zo optimaal mogelijke opstelling voor het bepalen van de locatie waar een voorwerp zich bevindt. Verder is ook de detectie van een verplaatsing belangrijk, dit gebeurt als de locatie van het voorwerp verandert. Met oog op deze probleemstelling luidt de onderzoeksvraag voor deze bachelorproef als volgt:
\begin{center}
	\textbf{Welke hardwareopstelling, bestaande uit RFID of BLE componenten, is optimaal voor de plaats- en verplaatsingsbepaling van een voorwerp binnen een gebouw.}
\end{center}
Deze vraag omvangt goed de essentie van de situatie die dient onderzocht te worden, namelijk de beste opstelling voor het volgen van een voorwerp binnen een gebouw, en dit met RFID of BLE technologie. Binnen de probleemstelling werd echter ook aangegeven dat binnen de scope van Polaris ook gps lokalisatie voor voorwerpen buiten het bedrijf aanwezig was. Locatiebepaling met behulp van gps is echter al goed ingeburgerd, waardoor er voldoende bronnen zijn en dit ook duidelijk en precies is. Door deze reden is er geen nut om dit ook te onderzoeken en wordt dit onderdeel buiten de scope van dit onderzoek gehouden. Plaatsbepalingen binnen zijn echter niet zo optimaal voor gps aangezien de meeste gebouwen/kantoren beschikken over verdiepingen (waar er meerdere locaties dezelfde geografische coördinaten hebben) en gps ook een bepaalde onzekerheid heeft op de meting.

\section{Onderzoeksdoelstelling}
\label{sec:onderzoeksdoelstelling}

Het hoofddoel van dit onderzoek is het vergelijken van een aantal opstellingen, met RFID of met BLE, en met verschillende concepten achter hun werking. De verschillende opstellingen met hun theoretische achtergrond volgen verder in dit verslag. Zoals de onderzoeksvraag echter al doet vermoeden is het doel niet louter een vergelijkende studie, aangezien het ook de bedoeling is dat er een optimale opstelling uit de bus komt. 
In praktijk zal er nooit een opstelling zijn die de beste is, aangezien er altijd een afweging zal zijn tussen de nauwkeurigheid en de kostprijs van de opstelling. Ook zullen sommige opstellingen beter zijn in bepaalde situaties dan andere. Wel zal het mogelijk zijn om totaal onpraktische en onnauwkeurige opstellingen uit te branden. De conclusie zal eerder een afweging geven tussen de verschillende opstellingen, maar zal zeker geen louter vergelijkende aard hebben.

\section{Opzet van dit onderzoek}
\label{sec:opzet-bachelorproef}

De rest van dit onderzoek is als volgt opgebouwd:

In Hoofdstuk~\ref{ch:literatuurstudie} wordt een overzicht gegeven van de theoretische achtergrond die nodig is voor het begrijpen van het onderzoek en de rest van dit onderzoek. Het bevat informatie over de werking van de RFID en BLE technologieën, definities van veelvoorkomende begrippen en diverse andere benodigde uitleg.

In Hoofdstuk~\ref{ch:opstellingen} worden de onderzochte opstellingen opgesomd, samen met een theoretische achtergrond. Het is een projectie van de kennis opgedaan tijdens de literatuurstudie op de te onderzoeken opstellingen.

In Hoofdstuk~\ref{ch:methodologie} wordt de methodologie toegelicht en worden de gebruikte onderzoekstechnieken besproken om een antwoord te kunnen formuleren op de onderzoeksvraag.

In Hoofdstuk~\ref{ch:testen} worden de uitgevoerde tests beschreven, de resultaten verwerkt en geconcludeerd.

In Hoofdstuk~\ref{ch:conclusie}, tenslotte, wordt eerst de voornaamste informatie gebundeld, waarna deze gebruikt wordt om een algemene conclusie te gegeven en een antwoord te formuleren op de onderzoeksvraag. Daarna wordt ook een aanzet gegeven voor toekomstig onderzoek binnen dit domein.